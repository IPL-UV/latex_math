    Sleek Template is a minimal collection of \LaTeX{} packages and settings that ease the writing of beautiful documents. While originally meant for theses, it is perfectly suitable for project reports, articles, syntheses, etc. -- with a few adjustments, like margins.

    It is composed of four separate packages -- \texttt{sleek}, \texttt{sleek-title}, \texttt{sleek-theorems} and \texttt{sleek-listings} -- each of which can be used individually.

    \begin{lstlisting}[style=latexFrameTB, caption={Example of Sleek Template packages usage.}, gobble=8]
        \usepackage[english]{babel}
        \usepackage[noheader]{packages/sleek}
        \usepackage{packages/sleek-title}
    \end{lstlisting}

    \blindfootnote{If you are a \LaTeX{} beginner consider the excellent \href{https://www.overleaf.com/learn}{Overleaf tutorial}. Also, there are a lot of symbols available in \LaTeX{} and, therefore, in this template. I recommend the use of \enquote{The Comprehensive \LaTeX{} Symbol List} \cite{pakin2020comprehensive} for searching symbols.}
